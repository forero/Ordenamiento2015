\documentclass[letterpaper,12pt,onecolumn]{article}
\usepackage[spanish]{babel}
\usepackage[latin1]{inputenc}
\usepackage[pdftex]{color,graphicx}
\usepackage{hyperref}
\setlength{\oddsidemargin}{0cm}
\setlength{\textwidth}{490pt}
\setlength{\topmargin}{-40pt}
\addtolength{\hoffset}{-0.3cm}
\addtolength{\textheight}{4cm}

\begin{document}
\pagestyle{empty}
\section*{{\Large{\sc Desarrollo Institucional}}}

Mi servicio al Desarrollo Institucional se ha dado en todos los
niveles. Desde la colaboraci\'on tareas internas al departamento de
F\'isica hasta la visibilidad de la Uniandes a trav\'es del liderazgo
en iniciativas internacionales. 

Como servicio al departamento de F\'isica he sido dos veces parte del
comit\'e que prepara y hace el ex\'amen de conocimientos para estudiantes
de doctorado.  Para el mismo Departamento y para la Facultad de
Ciencias he trabajado como miembro del comit\'e de temas
computacionales. Este comit\'e se cre\'o en el 2014 principalmente
para coordinar la creaci\'on de un Laboratorio de Computaci\'on de
Alto Rendimiento y administrar recursos materiales para la ense\~nanza
de temas computacionales. El comit\'e tiene un representante de cada
departamento de Ciencias e incluye tambi\'en al vicedecano de
investigaciones de ciencias, al decano de ciencias y a un
representante de DSIT. Los mayores \'exitos de gesti\'on de este
comit\'e han sido la habilitaci\'on de un nuevo espacio en el edificio
Q para el trabajo colaborativo en investigaci\'onn computacional, el
establecimiento de un nuevo cl\'uster de c\'omputo de alto rendimiento en
colaboraci\'on con DSIT y la contrataci\'on de un cient\'ifico que trabaja
como puente entre las necesidades de los investigadores y DSIT. Este
trabajo ha implicado principalmente el dise\~no de acuerdos, reglas de
juego y seguimiento al uso de recursos aprobados por Ciencias y DSIT.

A nivel de Bogot\'a le he dado visibilidad a la Universidad a trav\'es de
la organizaci\'on de eventos abiertos a toda la comunidad para
entablar un di\'alogo entre el Arte y la Ciencia. Durante el 2012
organizamos los Encuentros de Arte y Ciencias en colaboraci\'on con la
Universidad Nacional. En estos encuentros se reun\'ian por tres horas
dos cient\'iicos y dos artistas para exponer su trabajo e intercambiar
opiniones con el p\'ublico. Tres encuentros se hicieron en Uniandes
con apoyo del departamento de F\'isica y el departamento de Artes. 
En el 2013 con el apoyo ecoc\'omico de la Uni\'on Astron\'omica
Internacional empec\'e el proyecto Astronom\'ia Perif\'erica que busca
llevar el poder creativo de la astronom\'ia a trav\'es del arte a lugares
perif\'ericos de la ciudad. Esto lo hice en colaboraci\'on con amigos
artistas para hacer cuatro intervenciones diferentes. Estuvimos en El
Carmen, Soacha y Bel\'en. Este proyecto se reipiti\'o en el 2014 gracias a
una beca de Space Art del Planetario de Bogot\'a. Esta vez el proyecto
principal estaba liderado por un artista y dise\~nador, todo en
colaboraci\'onn conmigo,  un estudiante de maestr\'ia y una jefe de
laboratorio de Uniandes.
A nivel regional organic\'e  en Julio del 2013 en Uniandes un encuentro
cient\'ifico llamado Astronom\'ia en los Andes para reunir astr\'onomos
profesionales de Venezuela, Colombia, Ecuador, Per\'u, Bolivia y Chile
con el fin de explorar la posibilidad de establecer proyectos de
colaboraci\'o  en invesitigaci\'on, ense\~nanza y divulgaci\'on de la
astronom\'ia. Ese encuentro fu\'e un \'exito dados los proyectos que
se empezaron a desarrollar. El segundo encuentro ser\'a en Julio del
2015 de nuevo en Uniandes. 

A nivel internacional lo m\'as relevante ha pasado como resultado de estos
encuentros Astronom\'iaa en los Andes. Basados en los proyectos de
colaboraci\'o  que iniciamos, logramos que la Uni\'on Astron\'omica
Internacional apoyara la creaci\'on de una Oficina de Astronom\'ia
para el Desarrollo en el \'area Andina. Uniandes, el Parque Explora de
Medell\'in y la Sociedad Chilena de Astronom\'ia act\'uan como los
coordinadores de esta iniciativa y yo tengo el rol de coordinador
general. La gran ventaja de este apoyo institucional es que nos
abrir\'a las puertas para conseguir buena financiaci\'on para
implementar nuestros proyectos. 

El apoyo de la UAI viene con un financiamiento inicial simb\'olico de
5000 Euros. La firma que da inicio a esta Oficina Regional se har\'a a
finales de Julio del 2015 en la clausura del segundo Astronom\'ia en los Andes. 

  
A continuaci\'on encontrar\'an 
\begin{itemize}
\item  Una lista completa de charlas cient\'ificas y divulgativas que
  he dado durante mi tiempo en Uniandes. 
\item La propuesta completa para la creaci\'on de la Oficina Regional
  Andina de Astronom\'ia para el Desarrollo da la cual soy el
  coordinador general.
\end{itemize}

\end{document}


