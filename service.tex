\documentclass{report}
\begin{document}


\section{Desarrollo Institucional}

Como servicio al departamento de Física he sido dos veces parte del
comité que prepara y hace el exámen de conocimientos para estudiantes
de doctorado. 

Para el mismo Departamento y para la Facultad de Ciencias he trabajado
como miembro del comité de temas computacionales. Este comité se creó
en el 2014 principalmente para coordinar la creación de un Laboratorio
de Computación de Alto Rendimiento y administrar recursos materiales
para la enseñanza de temas computacionales. El comité tiene un
representante de cada departamento de Ciencias e incluye también al vicedecano
de investigaciones, al decano y un representante de DSIT. El mayor éxito de la
gestión de este comité ha sido la habilitación de un nuevo espacio en el edificio Q
para el trabajo colaborativo en investigación computacional, el
establecimiento de un nuevo clúster de cómputo de alto rendimiento en
colaboración con DSIT y la contratación de un científico que trabaja
como puente entre las necesidades de los investigadores y DSIT. Este
trabajo ha implicado principalmente el diseño de acuerdos, reglas de
juego y seguimiento al uso de recursos aprobados por Ciencias y DSIT.

A nivel de Bogotá le he dado visibilidad a la Universidad a través de
la organización de enventos que unen el Arte y la Ciencias. Durante el
2012 organizamos los Encuentros de Arte y Ciencias en colaboración con
la Universidad Nacional. En estos encuentros se reunían por tres horas
dos científicos y dos artistas para exponer su trabajo e intercambiar
opiniones con el público. Tres encuentros se hicieron en Uniandes con
apoyo del departamento de Física y el departamento de Artes.

En el 2013 con el apoyo económico de la Unión Astronómica
Internacional empecé el proyecto Astronomía Periférica que busca
llevar el poder creativo de la astronomía a través del arte a lugares
periféricos de la ciudad. Esto lo hice en colaboración con amigos
artistas para hacer 4 intervenciones diferentes. Estuvimos en El
Carmen, Soacha y Belén. Este proyecto se reipitió en el 2014 gracias a
una beca de Space Art del Planetario de Bogotá, esta vez el proyecto
principal estaba liderado por un artista y diseñador, todo en
colaboración con un estudiante de maestría y una jefe de laboratorio
de Uniandes.

A nivel regional organicé en Julio del 2013 en Uniandes un encuentro científico
llamado Astronomía en los Andes que reunió astrónomos profesionales de
Venezuela, Colombia, Ecuador, Peru, Bolivia y Chile con el fin de
explorar la posibilidad de establecer proyectos de colaboración en
invesitigación, enseñanza y divulgación de la astronomía. Ese
encuentro fué un éxito dados los proyectos que se empezaron a
desarrollar. El segundo encuentrp será en Julio del 2015 de nuevo en
Uniandes. 

A nivel internacional lo más relevante ha pasado como resultado de los
encuentros Astronomía en los Andes. Basados en los proyectos de
colaboración logramos que la Unión Astronómica Internacional apoyara
la creación de una Oficina de Astronomía para el Desarrollo en el área
Andina. Uniandes, el Parque Explora de Medellín y la Sociedad Chilena
de Astronomía actúan como los coordinadores de esta iniciativa y yo
tengo el rol de coordinador general. La gran ventaja de este apoyo
institucional es que nos abrirá las puertas para conseguir buena
financiación para implementar nuestros proyectos. El apoyo de la UAI
viene con un financiamiento inicial simbólico de 5000 Euros. La firma
que da inicio a esta Oficina Regional se hará a finales de Julio del
2015 en la clausura del segunod Astronomía en los Andes. 
 

\end{document}


