Investigación

En mi tiempo en el Departamento de Física he logrado mantener una alta productividad científica. He publicado 10 papers manteniendo colaboraciones con Estados Unidos, Alemania, Chile, Corea del Sur, España y Brasil. Una de esas publicaciones la hice con un estudiante de Maestría de Uniandes, otra con un estudiante maestría de la Universidad de Antioquia y otra con un estudiante de maestría que en Venezuela. Esta productividad ha ayudado para que el grupo de Investigación de Astrofísica pueda encontrarse ahora en clasificación A1 de COLCIENCIAS según los resultados publicados el 11 de marzo del 2015. Cuando entré al grupo la clasificación de COLCIENCIAS era C. 

Con mi inclusión en el grupo de Astrofísica también llega el primer estudiante de Doctorado del grupo, Felipe Gómez, quien se encuentra actualmente en una pasantía en Purdue después de haber ganado una convocatoria con COLCIENCIAS. Felipe está terminando su segundo año de estudios de doctorales y va a presentar su examen de conocimientos en Julio próximo.

Otra novedad es el grupo estudiantes de pregrado que han empezado a trabajar en temas de investigación conmigo. Todos han llegado desde 4 o 5 semestre de la carrera. Varios de ellos ya han presentado resultados en congresos internacionales y nacionales. Actualmente dos de ellos (Christian Poveda de octavo semestre y María Camila Remolina de sexto semestre) están a punto de enviar sus resultados a revistas internacionales. 

Mi trabajo de investigación ya tiene visibilidad a nivel nacional. En el 2012 y 2014 fuí invitado a ser parte del Comité Científico del Congreso Colombiano de Astronomía y Astrofísica. También he sido invitado para ser jurado de una tesis de maestría y una tesis de doctorado en la Universidad Nacional de Colombia. 

A nivel internacional mi actividad también es reconocida a través de invitaciones a ser referee de publicaciones internacionales (Astrophysical Journal, Monthly Notices of the Royal Astronomical Society, Journal of Cosmology and Astroparticle Physics) y una charla invitada en un congreso internacional de Cosmología en Corea del Sur. Como mencionaba al comienzo, mi record de publicaciones y proyectos se hacen en su totalidad con grupos que se encuentran por todo el mundo, principalmente en Europa, Estados Unidos y Corea del Sur.

Tal vez el logro más significativo de mi tiempo en Uniandes es la colaboración con el Dark Energy Spectroscopic Instrument (DESI). DESI es una colaboración internacional para hacer un mapa 3D de la distribución de galaxias para descubrir los orígenes de la expansión acelerada del Universo. Esta colaboración de cientos de científicos es liderada por UC Berkeley incluyendo al premio Nobel (Física, 2011) Saul Perlmutter. El proyecto está en su fase de construcción entre el 2014 y el 2018 para tomar datos entre el 2018 y el 2022. Ahora Uniandes está trabajando con esta colaboración, gracias a mis contactos en Berkeley, los contactos del grupo de Altas Energías en Fermilab (miembro de DESI) y el apoyo económico de Altas Energías para pagar pasantías de estudiantes de doctorado para este proyecto. La confirmación oficial de la pertenencia de Uniandes a esta colaboración debería confirmarse en el verano del 2015.



Enseñanza

Métodos Computacionales. 

AstroLunch.

Nuevo CBU.

Profesor en Escuelas Internacionales en Guatemala y en Ecuador.


Desarrollo Institucional.

Departamento de Física y Facultad de Ciencias. Como servicio al departamento de Física he sido dos veces parte del comité que prepara y hace el exámen de conocimientos. 
Para la Facultad de Ciencias he trabajado como miembro del comité de temas computacionales. Este comité se creó en el 2014 principalmente para coordinar la creación de un Laboratorio de Computación de Alto Rendimiento y administrar recursos materiales para la enseñanza de temas computacionales. El mayor éxito de nuestra gestión ha sido la habilitación de un nuevo espacio en el edificio Q para el trabajo colaborativo en investigación computacional, el establecimiento de un nuevo clúster de cómputo de alto rendimiento en colaboración con DSIT.

Visibilidad Local. Encuentros Arte y Ciencia. Astronomia Periferica.

Visibilidad Nacional. Comité Científico COCOA

Visibilidad Regional. Astronomía en los Andes I y II.

Visibilidad Internacional. Nodo Andino de la IAU. 



