\documentclass[letterpaper,12pt,onecolumn]{article}
\usepackage[spanish]{babel}
\usepackage[latin1]{inputenc}
\usepackage[pdftex]{color,graphicx}
\usepackage{hyperref}
\setlength{\oddsidemargin}{0cm}
\setlength{\textwidth}{490pt}
\setlength{\topmargin}{-40pt}
\addtolength{\hoffset}{-0.3cm}
\addtolength{\textheight}{4cm}

\begin{document}
\pagestyle{empty}
\section*{{\Large{\sc Resumen de actividades de investigaci\'on}}}

En mi tiempo en el Departamento de F\'isica he logrado una alta
productividad cient\'ifica. He publicado {\bf 10 papers} en colaboraci\'on
con cient\'ificos de Europa, Estados Unidos, Corea del Sur y Chile. Al
mismo tiempo dentro de Uniandes he logrado consolidar alrededor de mi
l\'inea de investigaci\'on ({\bf cosmolog\'ia computacional}) un excelente
grupo de colaboradores a nivel de pregrado, maestr\'ia, doctorado y
postdoctorado.  

Esta productividad y coherencia cient\'ifica han ayudado para que el grupo de
Investigaci\'on de Astrof\'isica pueda encontrarse ahora en {\bf clasificaci\'on
A1 de COLCIENCIAS} seg\'un los resultados publicados a comienzos de marzo del
2015. Cuando entr\'e al grupo nuestra clasificaci\'on de COLCIENCIAS era C.   

Conmigo llega {\bf el primer estudiante de Doctorado del grupo de Astrof\'isica}, Felipe
G\'omez, quien se encuentra actualmente en una pasant\'ia en Purdue
despu\'es de haber ganado una convocatoria con COLCIENCIAS. Felipe
est\'a terminando su segundo a\~no de estudios de doctorales y va a
presentar su examen de conocimientos en Julio pr\'oximo. {\bf Tambi\'en
trabajo y publico con estudiantes de maestr\'ia} de Uniandes (Juan
Nicolas Garavito, pr\'oximo a graduarse, admitido a doctorado en la
University of Arizona), UdeA (Sebasti\'an
Bustamante, quien empezar\'a en Heidelberg un doctorado en cotutela conmigo)
y Venezuela (Jose Hern\'andez, invitado a una pasant\'ia corta conmigo
en Uniandes, actualmente terminando su primer a\~no de doctorado en
Francia).   

Tambi\'en {\bf me siento muy orgulloso del grupo estudiantes de pregrado que
hacen investigaci\'on conmigo}. Todos han llegado desde tercer o cuarto
semestre de la carrera. Varios de ellos ya han presentado resultados
en congresos nacionales e internacionales. Actualmente dos de ellos
(Christian Poveda de octavo semestre y Mar\'ia Camila Remolina de
sexto semestre) est\'an preparando publicaciones sobre sus
resultados. Otros cuatro estudiantes est\'an en proceso de obtener sus
primeros resultados.  

Otro buen signo de la visibilidad de la investigaci\'on que vengo
haciendo es {\bf la llegada este semestre de Ver\'onica \'Arias como 
investigadora postdoctoral}. Ella llega a trabajar bajo mi
tutor\'ia despu\'es de obtener su doctorado en Alemania y haber pasado
dos a\~nos como postdoc en Australia. 

Mi trabajo de investigaci\'on ya tiene visibilidad a nivel nacional. En
el 2012 y 2014 he sido invitado a ser parte del Comit\'e Cient\'ifico del
Congreso Colombiano de Astronom\'ia y Astrof\'isica. Tambi\'en he sido
invitado para ser jurado de una tesis de maestr\'ia y una tesis
doctorado en la Universidad Nacional de Colombia.  A nivel
internacional mi actividad tambi\'en es reconocida a trav\'es de
invitaciones a ser referee de publicaciones internacionales
(Astrophysical Journal, Monthly Notices of the Royal Astronomical
Society, Journal of Cosmology and Astroparticle Physics) y una charla
invitada en un congreso internacional de Cosmolog\'ia en Corea del Sur.    


{\bf Otro logro significativo de mi tiempo en Uniandes ha sido
empezar a trabajar con el Dark Energy Spectroscopic Instrument (DESI)}. DESI
es una colaboraci\'on internacional que har\'a un mapa 3D de la
distribuci\'on de galaxias para descubrir los or\'igenes de la expansi\'on
acelerada del Universo. Esta colaboraci\'on de cientos de cient\'ificos es
liderada por UC Berkeley incluyendo al premio Nobel (F\'isica, 2011)
Saul Perlmutter. El proyecto estar\'a en su fase de construcci\'on entre el
2014 y el 2018, empezar\'a a tomar datos entre el 2018 y el 2022. Ahora
Uniandes est\'a trabajando con esta colaboraci\'on, gracias a mis
contactos en Berkeley, los contactos del grupo de Altas Energ\'ias en
Fermilab (otro miembro de DESI) y el apoyo econ\'omico del mismo grupo
de Altas Energ\'ias para pagar pasant\'ias de estudiantes de doctorado
para este proyecto. La confirmaci\'on oficial de la pertenencia de
Uniandes a DESI deber\'ia darse en el verano del 2015. En la lista de
proceedings y contribuciones a congresos he resaltado los resultados
en colaboraci\'on con cient\'ificos de DESI.

Como anexos podr\'an encontrar los siguientes documentos que
ampl\'ian los detalles del panorama que he dibujado hasta ahora.

\begin{itemize}
\item Lista completa de publicaciones en revistas internacionales
  con revisi\'on de pares. Incluyo un an\'alisis de citaciones.
\item Lista completa de publicaciones en proceedings y res\'umenes de
  contribuci\'on e encuentros internacionales importantes. Resalto los
  relacionados con DESI.
\item Lista de todas las propuestas de financiamiento que he enviado
  desde Uniandes.
\item Lista completa de estudiantes y postdocs en Uniandes que he guiado
  en labores de investigaci\'on. 
\item Los textos completos de todas las publicaciones que he hecho desde
  Uniandes.
\end{itemize}

\end{document}


