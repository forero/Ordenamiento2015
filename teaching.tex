\documentclass{report}
\begin{document}

\section{Enseñanza}


Mi labor principal de enseñanza ha sido en cursos de servicio del
departamento de Física (Física 1 y Física 2) que he dictado desde mi
primer semestre en la Universidad con un desempeño satisfactorio por
mi parte y la de los estudiantes.

Al interior de la Facultad de Ciencias uno de mis objetivos
principales ha sido fortalecer los cursos computacionales ofrecidos por
el departamento de Física. En mi tiempo en Uniandes viví la transición
hacia un pénsum nuevo para la carrera de Física. De 135 créditos
totales solamente 6 tenían que ver con cursos computacionales (APO y
Física Computacional). Actualmente hay dos cursos nuevos (Herramientas
Computacionales y Laboratorio de Métodos Computacionales) de 1 crédito
y el programa de Física Computacional se renovó para convertirse en
Métodos Computacionales. Yo tuve la fortuna de poder diseñar y
supervisar la implementación de los cursos nuevos así como de renovar
el curso de Métodos  Computacionales. Ahora mismo acaba de aprobarse
en el departamento de Física mi propuesta para una electiva llamada
Métodos Computacionales Avanzados que busca enseñarle a los
estudiantes como aprovechar los nuevos  recursos de Cómputo de Alto
Rendimiento con los que cuenta la Universidad.


El curso de Métodos Computacionales lo he podido dictar 4 veces en los
últimos dos años y medio. Todo el material que se he venido
desarrollando en colaboración con estudiantes, monitores y nuevos
profesores de la materia es público y accesible a través de un
repositorio de datos: github.com/ComputationalMethods.
Con este material estoy empezando a preparar un libro electrónico que
será editado por la Facultado de Ciencias. Un legado que quedará para
toda la comunidad hispanohablante interesada en tener acceso a un
curso moderno y actualizado sobre métodos computacionales.


Otra línea que ha guiado mi trabajo de enseñanza ha sido motivar a los estudiantes
de pregrado a hacer investigación desde los primeros semestres. Para
esto creé en el primer semestre del 2013 un espacio llamado astrolunch
los viernes de 1PM a 2PM donde nos reunimosa almorzar para hablar
sobre resultados recientes de investigación y donde además cada
estudiante tiene la posibiildad de iniciarse en la investigación a
través de pequeños proyectos. Astrolunch empezó con 3 estudiantes y
ahora es una materia de un crédito (Taller de Astronomía) que recibe
25 estudiantes cada semestre de todas las carreras y todos los
semestres. Los proyectos que se desarrollan los dirigen profesores y
estudiantes de posgrado del grupo de astronomía. El semestre cierra
con una sesi\'on de posters donde todos los participantes muestran sus
resultados. Algunos de los
proyectos de Astrolunch se han crecido para convertirse en resultados
publicables y, lo más importante en mi opinión, para motivar a los
estudiantes a pensar seriamente en lo que significa hacer investigación.


Un producto parcialmente derivado de Astrolunch es el nuevo CBU {\bf Astronomía
Popular} \footnote{Aprobado por el comit\'e de CBU de Ciencias e
  Ingenier\'ia a comienzos de Marzo del 2015. Se empezar\'a a ofrecer
  en el segundo semestre del 2015}. El origen del CBU son todas las
preguntas que han ido apareciendo en astrolunch donde la astronom\'ia
tiene alg\'un v\'inculo con elementos de la cultura popular. Temas del
curso incluyen: mitos de origen, fin del mundo, vida en otros planetas,
los agujeros negros como los muestran las películas, música con tema
espacial y lo que dice la ciencia sobre la astrología.

Mi labor docente también se ha extendido a escuelas
internacionales. En el 2012 fui profesor en una escuela del
PanAmerican Science Institute (PASI) que se realizó en Guatemala sobre
temas de Data Science. Yo dí un cursillo sobre cosmología
computacional y herramientas para hacer investigagación computacional
que se reproducible y abierta. A finales del 2014 fuí profesor en la
Primera Escuela Andina de Astronomía y Astrofísica que se hizo en
Quito. Allí dicté una serie de charlas sobre métodos computacionales y
uso de bases de datos públicas. Actualmente estoy preparando la
Escuela Andina de Cosmología en Uniandes que se desarrollará durante
todo el mes de Junio del 2015 con la asistencia de 3 profesores
internacionales (Alemania, Israel y Estados Unidos) y 25 estudiantes
de toda el área andina (Venezuela, Colombia, Ecuador, Perú, Bolivia y
Chile). 



\end{document}


