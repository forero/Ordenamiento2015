\documentclass[letterpaper,12pt,onecolumn]{article}
\usepackage[spanish]{babel}
\usepackage[latin1]{inputenc}
\usepackage[pdftex]{color,graphicx}
\usepackage{hyperref}
\setlength{\oddsidemargin}{0cm}
\setlength{\textwidth}{490pt}
\setlength{\topmargin}{-40pt}
\addtolength{\hoffset}{-0.3cm}
\addtolength{\textheight}{4cm}

\begin{document}
\pagestyle{empty}
\section*{{\Large{\sc Ense\~nanza}}}

Mis labores de ense\~nanza tienen tres hilos conductores: dar excelentes
cursos de servicio de F\'isica, renovar la ense\~nanza computacional
para estudiantes de F\'isica; y crear y compartir conocimiento
astron\'omico con estudiantes de pregrado.

Mi labor principal de ense\~nanza ha sido en cursos de servicio del
departamento de F\'isica (F\'isica 1 y F\'isica 2) que he dictado desde mi
primer semestre en la Universidad con un desempe\~no satisfactorio por
mi parte y la de los estudiantes.

Al interior de la Facultad de Ciencias uno de mis objetivos
principales ha sido fortalecer los cursos computacionales ofrecidos por
el departamento de F\'isica. En mi tiempo en Uniandes viv\'i la transici\'on
hacia un p\'ensum nuevo para la carrera de F\'isica. De 135 cr\'editos
totales solamente 6 ten\'ian que ver con cursos computacionales
(Algor\'itmica y Programaci\'on Orientada por Objetos y 
F\'isica Computacional). Actualmente hay dos cursos nuevos (Herramientas
Computacionales y Laboratorio de M\'etodos Computacionales) de 1 cr\'edito
 cada uno; el programa de F\'isica Computacional se renov\'o para convertirse en
M\'etodos Computacionales. Yo tuve la fortuna de poder dise\~nar y
supervisar la implementaci\'on de los cursos nuevos as\'i como de renovar
el curso de M\'etodos  Computacionales. Ahora mismo acaba de aprobarse
en el departamento de F\'isica mi propuesta para una electiva llamada
M\'etodos Computacionales Avanzados que busca ense\~narle a los
estudiantes como aprovechar los nuevos  recursos de C\'omputo de Alto
Rendimiento con los que cuenta la Universidad.


El curso de M\'etodos Computacionales lo he podido dictar 4 veces en los
\'ultimos dos a\~nos y medio. Todo el material que se he venido
desarrollando en colaboraci\'on con estudiantes, monitores y nuevos
profesores de la materia es accesible a trav\'es de un
repositorio p\'ublico: \url{https://github.com/forero/ComputationalMethods}.
Con este material estoy empezando a preparar un libro electr\'onico que
ser\'a editado por la Facultado de Ciencias. Un legado que quedar\'a para
toda la comunidad hispanohablante interesada en tener acceso a un
curso moderno y actualizado sobre m\'etodos computacionales.


Desde mi campo de investigaci\'on y ense\~nanza en temas
computacionales, tambi\'en me interesan los temas de innovaci\'on en
educaci\'on a trav\'es de tecnolog\'ia de la informaci\'on. He
asistido a eventos del centro de Innovaci\'on en Tecnolog\'ia y
Educaci\'on (Conecta-TE) de la Universidad. Esto tambi\'en me llev\'o
a acercarme a la startup colombiana de mayor crecimiento en temas
educaci\'on online: Platzi. Ellos tienen la infraestructura par dictar
cursos hasta 10mil asistentes online al mismo tiempo. Me interesaba
saber c\'omo se podr\'ia entrar en colaboraci\'on con ellos para
dictar un curso de astronom\'ia que fuera de alcance internacional. En
el proceso de conversaci\'on con ellos particip\'e en un programa que
ellos hacen en vivo (mi participaci\'on puede verse aqu\'i:
\url{https://www.youtube.com/watch?v=6ApsFJPkXj4}). Actualmente la
gente de Platzi est\'a en una aceledora de startups en Silicon Valley
y nuestro proyecto de colaboraci\'on est\'a congelado.

Otra l\'inea que ha guiado mi trabajo de ense\~nanza ha sido motivar a los estudiantes de pregrado a hacer investigaci\'on desde los primeros semestres. Para
esto cre\'e en el primer semestre del 2013 un espacio llamado Astrolunch
los viernes de 1PM a 2PM donde nos reunimos a almorzar juntos para hablar
sobre resultados recientes de investigaci\'on en astronom\'ia,
astrof\'isica y ciencias del espacio donde adem\'s cada
estudiante tiene la posibiildad de iniciarse en la investigaci\'on a
trav\'es de peque\~nos proyectos. Astrolunch empez\'o con 3 estudiantes y
ahora es una materia de un cr\'edito (Taller de Astronom\'ia) que recibe
25 estudiantes cada semestre de todas las carreras y todos los
semestres. Los proyectos que se desarrollan los dirigen profesores y
estudiantes de posgrado del grupo de astronom\'ia. El semestre cierra
con una sesi\'on de posters donde todos los participantes muestran sus
resultados. Algunos de los proyectos de Astrolunch se han crecido para
convertirse en resultados publicables y, lo m\'as importante en mi
opini\'on, para motivar a los estudiantes a pensar seriamente en lo
que significa hacer investigaci\' n. 


Un producto parcialmente derivado de Astrolunch es el nuevo CBU {\bf
  Astronom\'ia Popular} \footnote{Aprobado por el comit\'e de CBU de Ciencias e
  Ingenier\'ia a comienzos de Marzo del 2015. Se empezar\'a a ofrecer
  en el segundo semestre del 2015}. El origen del CBU son todas las
preguntas que han ido apareciendo en astrolunch donde la astronom\'ia
tiene alg\'un v\'inculo con elementos de la cultura popular. Temas del
curso incluyen: mitos de origen, fin del mundo, vida en otros planetas,
los agujeros negros como los muestran las pel\'iculas, m\'usica con tema
espacial y lo que dice la ciencia sobre la astrolog\'ia.

Mi labor docente tambi\'en se ha extendido a escuelas
internacionales. En el 2013 fui profesor en una escuela del
PanAmerican Advanced Studies Institutes (PASI) que se realiz\'o en
Guatemala sobre temas de Data Science \footnote{Methods in
  Computation-Based
  Discovery. \url{http://ichass.illinois.edu/index.php/i-chass-and-oasartca-release-pasi-program-description/}}. D\'i
un cursillo sobre cosmolog\'ia computacional y herramientas para hacer
investigagaci\'on computacional que se reproducible y abierta. A
finales del 2014 fui profesor en la Primera Escuela Andina de Astronom\'ia y
Astrof\'isica \footnote{Primera Escuela Andina de Astronom\'ia y
  Astrof\'isica \url{http://departamentodefisicaeventos.epn.edu.ec/index.php?option=com_content&view=article&id=92&Itemid=478}}
que se hizo en Quito. All\'i dict\'e una serie de charlas sobre
m\'etodos computacionales y uso de bases de  p\'ublicas. Actualmente
estoy preparando la Escuela Andina de Cosmolog\'ia en
Uniandes \footnote{Escuela Andina de Cosmolog\'ia \url{http://forero.github.io/AndeanCosmologySchool/}}
que se desarrollar\'a durante todo el mes de Junio del 2015 con la
asistencia de 3 profesores internacionales (Alemania, Israel y Estados
Unidos) y 25 estudiantes de toda el \'area andina (Venezuela,
Colombia, Ecuador, Per\'u, Bolivia y Chile). Durante esta escuela
vamos a estrenar en un curso de ense\~nanza los nuevos recursos
computacionales de alto rendimiento de la Universidad.


Para poder dar una perspectiva m\'as clara sobre mis actividades de
ense\~nanza, a continuaci\'on podr\'an encontrar los siguientes documentos.
\begin{itemize}
\item Una lista completa de cursos que he dictado en Uniandes con
  informaci\'on sobre las calificaciones que he recibido. 
\item El programa del nuevo CBU Astronom\'ia Popular.
\item El programa del nuevo curso de Herramientas Computacionales.
\item El programa renovado de M\'etodos Computacionales tal como lo
  dict\'e la \'ultima vez.
\item El programa del nuevo curso electivo M\'etodos Computacionales
  Avanzados. 
\end{itemize}


\end{document}


